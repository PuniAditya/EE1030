\documentclass[xournal]{IEEEtran}
\usepackage[a5paper, margin=10mm, onecolumn]{geometry}
\usepackage{amsmath,amssymb,amsfonts,amsthm}
\usepackage{mathtools}
\usepackage{gvv-book}
\usepackage{gvv}
\usepackage{hyperref}

\begin{document}

\title{2.10.85}
\author{Puni Aditya - EE25BTECH11046}
\maketitle

\textbf{Question:}
Let $\vec{P}$ be the plane $3x + 2y + 3z = 16$ and let 
$S : \alpha \hat{i} + \beta \hat{j} + \gamma \hat{k},\text{ where } \alpha + \beta + \gamma = 7$
and the distance of $\brak{\alpha, \beta, \gamma}$ from the plane is $\frac{2}{\sqrt{22}}$.
Let $\vec{u}, \vec{v}, \vec{w}$ be three distinct vectors in $S$ such that $|\vec{u} - \vec{v}| = |\vec{v} - \vec{w}| = |\vec{w} - \vec{u}|$. Let $V$ be the volume of the parallelepiped determined by vectors $\vec{u}, \vec{v}, \vec{w}$. Then the value of $\frac{80}{3} V$ is
\rule{1cm}{0.1pt}.

\textbf{Solution:}

Let the matrix of vectors be $\vec{A} = \myvec{\vec{u} & \vec{v} & \vec{w}}$. The volume $V = \abs{\mydet{\vec{A}}}$.
Let $\vec{x} = \myvec{1 \\ 1 \\ 1}$. The condition that the points lie on the plane $\alpha+\beta+\gamma=k_1$ where $k_1=7$ is:
\begin{align}
    \vec{A}^\top\vec{x} &= k_1\vec{x} \label{eq:eigen}
\end{align}
So, $k_1$ is an eigenvalue of $\vec{A}^\top$ and hence an eigenvalue of $\vec{A}$. \\
The vectors representing the sides of the equilateral triangle are
\begin{align}
    \vec{u} - \vec{v} &= \vec{A}\vec{c_1} \label{eq:1} \\
    \vec{v} - \vec{w} &= \vec{A}\vec{c_2} \label{eq:2} \\
    \vec{w} - \vec{u} &= \vec{A}\vec{c_3} \label{eq:3}
\end{align}
where $\vec{c_1} = \myvec{-1 \\ 1 \\ 0}, \vec{c_2} = \myvec{0 \\ 1 \\ -1}, \vec{c_3} = \myvec{1 \\ 0 \\ -1}$. \\
The condition $\norm{\vec{u} - \vec{v}} = \norm{\vec{v} - \vec{w}} = \norm{\vec{w} - \vec{u}} = L$ implies:
\begin{align}
    \brak{\vec{A}\vec{c_1}}^\top\brak{\vec{A}\vec{c_1}} = \brak{\vec{A}\vec{c_2}}^\top\brak{\vec{A}\vec{c_2}} = \brak{\vec{A}\vec{c_3}}^\top\brak{\vec{A}\vec{c_3}} = L^2 \label{eq:gram}
\end{align}
\begin{align}
    V^2 = \mydet{\vec{G}} 
\end{align}
where the Gram matrix is $\vec{G} = \vec{A}^\top\vec{A}$. From \eqref{eq:eigen}, we find an eigenvector of $\vec{G}$:
\begin{align}
    \vec{G}\vec{x} &= \vec{A}^\top\vec{A}\vec{x} \\
    &= \vec{A}^\top\brak{k_1\vec{x}} \\
    &= k_1\brak{\vec{A}^\top\vec{x}} \\
    &= k_1\brak{k_1\vec{x}} \\
    &= k_1^2\vec{x}
\end{align}
So, $\lambda_1 = k_1^2$ is an eigenvalue of $\vec{G}$ with eigenvector $\vec{x}$. \\
\eqref{eq:gram} can be written in terms of $\vec{G}$ as
\begin{align}
    \vec{c_i}^\top \vec{G} \vec{c_i} = L^2 \text{ for } i=1,2,3
\end{align}
The vectors $\vec{c_i}$ are orthogonal to $\vec{x}$ because $\vec{x}^\top\vec{c_i}=0$ and lie in a 2D subspace $W$. \\
Since $\vec{G}$ is symmetric, its other two eigenvectors, $\vec{e_2}, \vec{e_3}$, span $W$. \\
Let their eigenvalues be $\lambda_2, \lambda_3$. \\
$\because \vec{c_1}^\top\vec{G}\vec{c_1} = L^2\text{ and }\vec{c_2}^\top\vec{G}\vec{c_2} = L^2; \norm{\vec{c_1}}^2 = 2\text{ and }\norm{\vec{c_2}}^2 = 2$, the quadratic form defined by $\vec{G}$ is constant on a circle (The ellipse becomes a circle due to equal distance of the points from centre) in the subspace $W$. \\
This requires the eigenvalues corresponding to this subspace to be equal: $\lambda_2 = \lambda_3 = \lambda$.
Therefore, for any vector $\vec{w} \in W$, $\vec{w}^\top\vec{G}\vec{w} = \lambda\norm{\vec{w}}^2$.
Using $\vec{c_1}$:
\begin{align}
    \vec{c_1}^\top\vec{G}\vec{c_1} &= \lambda\norm{\vec{c_1}}^2 \\
    L^2 &= \lambda\brak{2} \\
    \lambda &= \frac{L^2}{2}
\end{align}
The eigenvalues of $\vec{G}$ are $k_1^2, \frac{L^2}{2}, \frac{L^2}{2}$. The determinant of $\vec{G}$ is the product of its eigenvalues:
\begin{align}
    \mydet{\vec{G}} = k_1^2 \brak{\frac{L^2}{2}} \brak{\frac{L^2}{2}} = \frac{k_1^2 L^4}{4}
\end{align}
The volume is 
\begin{align}
    V = \sqrt{\mydet{\vec{G}}} = \frac{k_1 L^2}{2}
\end{align}
\begin{align}
    \vec{P}: 3x+2y+3z=16 \implies \vec{n} = \myvec{3 \\ 2 \\ 3}, k_2=16 \\
    x+y+z=7 \implies \vec{n_1} = \myvec{1 \\ 1 \\ 1}, k_1=7 \\
\end{align}
The distance condition gives two planes $\vec{P_1}: 3x+2y+3z=18$ and $\vec{P_2}: 3x+2y+3z=14$. \\
Let altitude of the triangle be $h_T$
\begin{align}
    h_T &= \frac{d_{P_1P_2}}{\sin\theta},\text{ where }\theta\text{ is the angle between the plane normals.} \\
    d_{P_1P_2} &= \frac{\abs{18-14}}{\norm{\vec{n}}} = \frac{4}{\sqrt{22}} \\
    \sin\theta &= \sqrt{1 - \left(\frac{\vec{n}^\top\vec{n_1}}{\norm{\vec{n}}\norm{\vec{n_1}}}\right)^2} = \sqrt{1 - \left(\frac{8}{\sqrt{22}\sqrt{3}}\right)^2} = \frac{1}{\sqrt{33}}
\end{align}
\begin{align}
    h_T &= \frac{4/\sqrt{22}}{1/\sqrt{33}} = 4\sqrt{\frac{3}{2}} \\
    L^2 &= \left(\frac{2}{\sqrt{3}}h_T\right)^2 = \frac{4}{3}\left(16 \times \frac{3}{2}\right) = 32 \\
    V &= \frac{\abs{7}\brak{32}}{2} = 112
\end{align}
\begin{align}
    \therefore \frac{80}{3}V &= \frac{80}{3}\brak{112} = \frac{8960}{3}
\end{align}

\begin{figure}[h!]
	\centering
	\includegraphics[width=\columnwidth]{figs/plot_p.jpg}
	\caption*{Plot}
	\label{fig:fig}
\end{figure}

\end{document}

\documentclass[journal]{IEEEtran}
\usepackage[a5paper, margin=10mm, onecolumn]{geometry}
\usepackage{amsmath,amssymb,amsfonts,amsthm}
\usepackage{gvv-book}
\usepackage{gvv}
\usepackage{hyperref}

\begin{document}

\title{10.7.75}
\author{Puni Aditya - EE25BTECH11046}
\maketitle

\textbf{Question:}
Find the equations of tangents drawn from origin to the circle $x^2+y^2-2rx-2hy+h^2=0$,are
\begin{multicols}{2}
\begin{enumerate}
    \item $x=0$
    \item $y=0$
    \item $\brak{h^2-r^2}x-2rhy=0$
    \item $\brak{h^2-r^2}x+2rhy=0$
\end{enumerate}
\end{multicols}

\textbf{Solution:}

A general conic section is described by the equation
\begin{align}
    \vec{x}^\top \vec{V} \vec{x} + 2\vec{u}^\top \vec{x} + f = 0
\end{align}
where $\vec{V}$ is a symmetric matrix. A line passing through a point $\vec{h}$ and having unit direction vector $\vec{m}$ is
\begin{align}
    \vec{x} = \vec{h} + k\vec{m}
\end{align}
Substitute the line equation into the conic to find points of intersection.
\begin{align}
    k^2\brak{\vec{m}^\top\vec{V}\vec{m}} + 2k\brak{\vec{m}^\top\vec{V}\vec{h} + \vec{u}^\top\vec{m}} + \brak{\vec{h}^\top\vec{V}\vec{h} + 2\vec{u}^\top\vec{h} + f} = 0
\end{align}
For the line to be tangent to the conic, the discriminant of quadratic in $k$ must be zero. \\
Let $g\brak{\vec{h}} = \vec{h}^\top\vec{V}\vec{h} + 2\vec{u}^\top\vec{h} + f$ be the value of the conic expression at the point $\vec{h}$.
\begin{align}
    \brak{\vec{m}^\top\vec{V}\vec{h} + \vec{u}^\top\vec{m}}^2 - \brak{\vec{m}^\top\vec{V}\vec{m}}\brak{g\brak{\vec{h}}} = 0
\end{align}
\begin{align}
    \brak{\vec{m}^\top \brak{\vec{V}\vec{h} + \vec{u}}}^2 - g\brak{\vec{h}}\brak{\vec{m}^\top\vec{V}\vec{m}} &= 0 \\
    \vec{m}^\top \brak{\brak{\vec{V}\vec{h} + \vec{u}}\brak{\vec{V}\vec{h} + \vec{u}}^\top} \vec{m} - \vec{m}^\top \brak{g\brak{\vec{h}}\vec{V}} \vec{m} &= 0 \\
    \vec{m}^\top \sbrak{\brak{\vec{V}\vec{h} + \vec{u}}\brak{\vec{V}\vec{h} + \vec{u}}^\top - \brak{\vec{h}^\top\vec{V}\vec{h} + 2\vec{u}^\top\vec{h} + f}\vec{V}} \vec{m} &= 0
\end{align}
\begin{align}
    \text{Let }\vec{\Sigma} = \brak{\vec{V}\vec{h} + \vec{u}}\brak{\vec{V}\vec{h} + \vec{u}}^\top - \brak{\vec{h}^\top\vec{V}\vec{h} + 2\vec{u}^\top\vec{h} + f}\vec{V} \label{eq:sigma_notation}
\end{align}
\begin{align}
    \vec{m}^\top \vec{\Sigma} \vec{m} = 0 \label{eq:sigma}
\end{align}
This is the general equation for the directions of tangents from an arbitrary point $\vec{h}$.

For $\vec{h} = \vec{0}$,
\begin{align}
    \vec{\Sigma} = \brak{\vec{V}\vec{0} + \vec{u}}\brak{\vec{V}\vec{0} + \vec{u}}^\top - \brak{\vec{0}^\top\vec{V}\vec{0} + 2\vec{u}^\top\vec{0} + f}\vec{V} = \vec{u}\vec{u}^\top - f\vec{V}
\end{align}

To solve this, the symmetric matrix $\vec{\Sigma}$ is diagonalized. The eigendecomposition of $\vec{\Sigma}$ is $\vec{\Sigma} = \vec{P}\vec{D}\vec{P}^\top$, where: \\
$\vec{D}$ is a diagonal matrix with the eigenvalues of $\vec{\Sigma}$ on its diagonal
\begin{align}
    \vec{D} = \myvec{\lambda_1 & 0 \\ 0 & \lambda_2} \label{eq:1}
\end{align}
$\vec{P}$ is an orthogonal matrix whose columns are the corresponding orthonormal eigenvectors. So, $\vec{P}^\top \vec{P} = \vec{P}\vec{P}^\top = 1$.
\begin{align}
    \vec{P} = \myvec{\vec{v_1} & \vec{v_2}} \label{eq:2}
\end{align}
Substitute \eqref{eq:1} and \eqref{eq:2} in \eqref{eq:sigma}
\begin{align}
    \vec{m}^\top (\vec{P}\vec{D}\vec{P}^\top) \vec{m} = 0
\end{align}
\begin{align}
    (\vec{m}^\top \vec{P})\vec{D}(\vec{P}^\top \vec{m}) = 0
\end{align}
Let $\vec{y} = \vec{P}^\top \vec{m} = \myvec{y_1 \\ y_2} $.
\begin{align}
    \vec{y}^\top \vec{D} \vec{y} = 0
\end{align}
\begin{align}
    \myvec{y_1 & y_2} \myvec{\lambda_1 & 0 \\ 0 & \lambda_2} \myvec{y_1 \\ y_2} &= 0 \\
    \myvec{y_1 & y_2} \myvec{\lambda_1 y_1 \\ \lambda_2 y_2} &= 0 \\
    \lambda_1 y_1^2 + \lambda_2 y_2^2 &= 0 \label{eq:e1}
\end{align}
Since $\vec{P}$ is orthogonal,
\begin{align}
    \vec{y}^\top\vec{y} &= \vec{m}^\top\vec{P}\vec{P}^\top\vec{m} \\
    &= \vec{m}^\top\vec{I}\vec{m} \\
    &= \vec{m}^\top\vec{m} \\
    &= 1
\end{align}
\begin{align}
    \implies y_1^2 + y_2^2 = 1 \label{eq:e2}
\end{align}
From the pair of equations \eqref{eq:e1} and \eqref{eq:e2},
\begin{align}
    y_1^2 = \frac{-\lambda_2}{\lambda_1-\lambda_2},\text{ }y_2^2 = \frac{\lambda_1}{\lambda_1-\lambda_2}
\end{align}
\begin{align}
    \vec{y} &= \vec{P}^\top \vec{m} \\
    \vec{P}\vec{P}^\top\vec{m} &= \vec{P}\vec{y} \\
    \vec{m} &= \vec{P}\vec{y} \label{eq:equation}
\end{align}

For the circle $x^2+y^2-2rx-2hy+h^2=0$,
\begin{align}
    \vec{A} = \myvec{1 & 0 \\ 0 & 1},\text{ }\vec{u} = \myvec{-r \\ -h},\text{ }f = h^2
\end{align}
From \eqref{eq:sigma_notation},
\begin{align}
    \vec{\Sigma} = \myvec{-r \\ -h}\myvec{-r & -h} - h^2\myvec{1 & 0 \\ 0 & 1} = \myvec{r^2-h^2 & rh \\ rh & 0}
\end{align}
The characteristic equation is $\mydet{\vec{\Sigma}-\lambda\vec{I}}=0$. \\
Let $\lambda^2 + a_1 \lambda + a_2 = 0$. Using Faddeev-Leverrier Method,
\begin{align}
    \vec{B_1} &= \vec{A} \\
    a_1 &= - tr\brak{\vec{B_1}} = -\brak{r^2-h^2}
\end{align}
\begin{align}
    \vec{B_2} &= \vec{A}\brak{\vec{B_1} + a_1\vec{I}} \\
    \vec{B_2} &= \myvec{r^2h^2 & 2rh\brak{r^2-h^2} \\ 0 & r^2h^2} \\
    a_2 &= -\frac{1}{2}tr\brak{\vec{B_2}} \\
    a_2 &= -r^2h^2
\end{align}
So, $\lambda^2 - \brak{r^2-h^2}\lambda - r^2h^2 = 0$, giving the eigenvalues $\lambda_1 = r^2$ and $\lambda_2 = -h^2$.
\begin{align}
    \vec{P} = \frac{1}{\sqrt{r^2+h^2}}\myvec{r & h \\ h & -r}
\end{align}
\begin{align}
    y_1^2 = \frac{-\brak{-h^2}}{r^2-\brak{-h^2}} = \frac{h^2}{r^2+h^2} \implies y_1 = \pm\frac{h}{\sqrt{r^2+h^2}} \\
    y_2^2 = \frac{r^2}{r^2-\brak{-h^2}} = \frac{r^2}{r^2+h^2} \implies y_2 = \pm\frac{r}{\sqrt{r^2+h^2}}
\end{align}
Using \eqref{eq:equation}, $\vec{m} = y_1\vec{v_1}+y_2\vec{v_2}$.
\begin{align}
    \vec{m_1} &\propto h\vec{v_1} - r\vec{v_2} \propto h\myvec{r \\ h} - r\myvec{h \\ -r} = \myvec{0 \\ h^2+r^2} \propto \myvec{0 \\ 1} \\
    \vec{m_2} &\propto h\vec{v_1} + r\vec{v_2} \propto h\myvec{r \\ h} + r\myvec{h \\ -r} = \myvec{2rh \\ h^2-r^2}
\end{align}
For $\vec{m_1} \propto \myvec{0 \\ 1}$: $1 \times x - 0 \times y = 0 \implies x = 0$. This is option \textbf{1}. \\
For $\vec{m_2} \propto \myvec{2rh \\ h^2-r^2}$: $\brak{h^2-r^2}x - \brak{2rh}y = 0$. This is option \textbf{3}.

\end{document}
